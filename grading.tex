\documentclass[prb,papersize=a4paper,notitlepage]{revtex4-1}%
\usepackage{hyperref}
\usepackage{enumitem}
\usepackage{nicefrac}
\usepackage{amsmath}
\usepackage{graphicx}
\usepackage{amsfonts}
\usepackage{physics}
\usepackage{amssymb}
\usepackage{bm}
\usepackage[utf8]{inputenc}
\usepackage[russian]{babel}
\usepackage{listings}


\begin{document}

\title{Вычислительная физика, Осень 2020 ВШЭ. Критерии оценки}
\maketitle
Итоговая оценка за курс определяется по формуле $$0.6\;\textrm{дз} + 0.1\;\textrm{активность} + 0.3\;\textrm{экзамен}.$$

Оценка за дз собирается из сдачи задач (теоретических и практических задач из заданий и технических упражнений по numpy). В течение семестра ожидается 9 заданий, максимально возможный балл по которым составит около 540 баллов (эти баллы указаны в условии каждой задачи). Дополнительные баллы можно заработать, сдавая $\star$ и $\star\star$ упражнения из списка \href{https://github.com/rougier/numpy-100/blob/master/100_Numpy_exercises.md}{100 Numpy exercises}. Эти упражнения оцениваются по формуле $\star = 1.5 \;\textrm{балла}$, так что максимально возможный балл за эти упражнения составляет около 140 баллов. Итоговая оценка за дз получается по формуле $$\textrm{дз} = 10\tanh\frac{S}{180},$$ где $S$ -- сумма всех баллов.

Оценка за активность ставится семинаристом (субъективно) в качестве оценки участия в семинарах (вопросы, найденные ошибки и выходы к доске).
\end{document}