\documentclass{article}
\usepackage[russian]{babel}
\usepackage[utf8]{inputenc}

\newcounter{eee}

\begin{document}

\section*{Вопросы по выбору}


Предложенные задачи заметно различаются по сложности. Для каждой задачи указан максимальный балл, который можно получить за идеальное решение.



\subsection*{Задачи на 10 баллов.}

\begin{enumerate}

\item Вычислить плотность энергии идеального ферми-газа в как функцию температуры в $d$ измерениях. 

\textit{Указание} Под плотностью энергии будем понимать отношение $Е/N$, где $N$ это число частиц. В безразмерных единицах вычислять нужно отношение $E/N\epsilon_F$, где $\epsilon_F$ обозначена энергия Ферми. Аналогично, обезразмеренная температура это отношение $T/\epsilon_F$. Результатом решения задачи будет вычисленный график зависимости $E/N\epsilon_F$ от $T/\epsilon_F$. Требуется также сравнить численные результаты с поведением в предельных случаях $T \ll \epsilon_F$ (вырожденный газ) и $T \gg \epsilon_F$ (классический газ + вириальная поправка).

Литература: Ландау-Лифшиц 5 том ; Fetter-Valecka, Quantum many-body theory, главы 1-2.

\begin{enumerate}
\item Для $d = 3$
\item Для $d = 2$
\item Для $d = 1$
\end{enumerate}

\item Вычислить плотность энергии и теплоемкость идеального Бозе-газа в зависмости от температуры в $d$ измерениях. 

\begin{enumerate}
\item Для $d=3$. Особое внимание обратить на окрестность температуры бозе-конденсации.
\item Для $d=2$
\item Для $d=1$
\end{enumerate}



\setcounter{eee}{\value{enumi}}
\end{enumerate}


\subsection*{Задачи на 9 баллов.}

\begin{enumerate}
\setcounter{enumi}{\value{eee}}  % restore the numbering


\item Рассмотрим изотермы реального  на плоскости $p$-$V$. При температурах ниже критической на изотермах появляется область сосуществования жидкость-газ, и зависимость $p(V)$ становится немонотонной. Фактически же давление остается постоянным во всей области сосуществования, причем положение горизонтального участка изотермы определяется правилом Максвелла и геометрически находится из равенства площадей участков изотермы. Построить изотермы газа с учетом конструкции Максвелла.
\textit{Указание: Первым действием найти критические параметры и переписать уравнение состояния в безразмерных единицах.}

\begin{enumerate}

\item Уравнение состояния ван-дер-Ваальса
$$
p = \frac{RT}{V - b} - \frac{a}{V^2} \;.
$$

\item Уравнение состояния Бертло

$$
p = \frac{RT}{V - b} - \frac{a}{T V^2} \;.
$$


\item  I уравнение Дитеричи

$$
p = \frac{RT}{V - b}  \exp{\left(-\frac{a}{R T V}\right)} \;.
$$


\item II уравнение Дитеричи

$$
p = \frac{RT}{V - b} - \frac{a}{V^{5/3}} \;.
$$


\item Уравнение состояния газа в виде вириального разложения до третьего порядка

$$
p = \frac{RT}{V} \left( 1 + \frac{B_2}{V} + \frac{B_3}{V^2} \right)\;.
$$


\end{enumerate}


\item Найти уровни энергии и волновые функции частицы в трехмерной прямоугольной яме конечной глубины  с угловым моментом $l$.

\item Рассмотрим частицу в одномерной прямоугольной яме ширины $a$. Добавим потенциал вида $V_0 x (x-a)$. Найти уровни энергии  и волновые функции, используя в качестве невозмущенного базиса состояния частицы в прямоугольной яме при $V_0 = 0$. Сравнить с результатами теории возмущений.



\end{enumerate}

\end{document}
