\documentclass{article}
\usepackage{amsmath}
\usepackage[russian]{babel}
\usepackage[utf8]{inputenc}

\newcounter{eee}

\begin{document}

\section*{Вопросы по выбору}


\begin{enumerate}

\item Вычислить плотность энергии идеального ферми-газа в как функцию 
температуры в $d$ измерениях.

\textit{Указание} Под плотностью энергии будем понимать отношение $Е/N$, где $N$ это число частиц. В безразмерных единицах вычислять нужно отношение $E/N\epsilon_F$, где $\epsilon_F$ обозначена энергия Ферми. Аналогично, обезразмеренная температура это отношение $T/\epsilon_F$. Результатом решения задачи будет вычисленный график зависимости $E/N\epsilon_F$ от $T/\epsilon_F$. Требуется также сравнить численные результаты с поведением в предельных случаях $T \ll \epsilon_F$ (вырожденный газ) и $T \gg \epsilon_F$ (классический газ + вириальная поправка).

Литература: Ландау-Лифшиц 5 том ; Fetter-Valecka, Quantum many-body theory, главы 1-2.

\begin{enumerate}
\item Для $d = 3$
\item Для $d = 2$
\item Для $d = 1$
\end{enumerate}

\item Вычислить плотность энергии и теплоемкость идеального Бозе-газа в
зависмости от температуры в $d$ измерениях.

\begin{enumerate}
\item Для $d=3$. Особое внимание обратить на окрестность температуры бозе-конденсации.
\item Для $d=2$
\item Для $d=1$
\end{enumerate}

\item Построить солитонное решение стационарного уравнения Кортевега-де-Вриза.
Написать программу, моделирующую динамику начально-граничной задачи с
несколькими солитонами при $t=0$.

\item Построить солитонное решение одномерного уравнения синус-Гордона (кинк).
Написать программу, моделирующую динамику двух кинков для различных начальных
условий.

\item Рассмотрим изотермы реального  на плоскости $p$-$V$. При температурах ниже критической на изотермах появляется область сосуществования жидкость-газ, и зависимость $p(V)$ становится немонотонной. Фактически же давление остается постоянным во всей области сосуществования, причем положение горизонтального участка изотермы определяется правилом Максвелла и геометрически находится из равенства площадей участков изотермы. Построить изотермы газа с учетом конструкции Максвелла.
\textit{Указание: Первым действием найти критические параметры и переписать уравнение состояния в безразмерных единицах.}

\begin{enumerate}

\item Уравнение состояния ван-дер-Ваальса
$$
p = \frac{RT}{V - b} - \frac{a}{V^2} \;.
$$

\item Уравнение состояния Бертло

$$
p = \frac{RT}{V - b} - \frac{a}{T V^2} \;.
$$


\item  I уравнение Дитеричи

$$
p = \frac{RT}{V - b}  \exp{\left(-\frac{a}{R T V}\right)} \;.
$$


\item II уравнение Дитеричи

$$
p = \frac{RT}{V - b} - \frac{a}{V^{5/3}} \;.
$$


\item Уравнение состояния газа в виде вириального разложения до третьего порядка

$$
p = \frac{RT}{V} \left( 1 + \frac{B_2}{V} + \frac{B_3}{V^2} \right)\;.
$$


\end{enumerate}


\item Найти уровни энергии и волновые функции частицы в трехмерной прямоугольной яме конечной глубины  с угловым моментом $l$.

\item Рассмотрим частицу в одномерной прямоугольной яме ширины $a$. Добавим потенциал вида $V_0 x (x-a)$. Найти уровни энергии  и волновые функции, используя в качестве невозмущенного базиса состояния частицы в прямоугольной яме при $V_0 = 0$. Сравнить с результатами теории возмущений.


\item Упрощенно промоделировать движение планет солнечной системы, учитывая
взаимное притяжение всех тел. Считать систему двумерной (рассмотреть плоскость
вращения планет вокруг Солнца), а планеты и Солнце материальными точками.
Считать, что в начальный момент времени все планеты находятся на одной линии на
соответствующем их реальным орбитам радиальном расстоянии и имеют наблюдаемые
орбитальные скорости. Провести моделирование на таком временном интервале, чтобы
все планеты совершили хотя бы один полный оборот.

\item Упрощенно промоделировать движение системы Солнце-Земля-Луна, учитывая
взаимное притяжение всех тел. Считать систему двумерной (рассмотреть плоскость
вращения планет вокруг Солнца), а космические тела материальными точками.
Считать, что в начальный момент времени Земля и Луна лежат на одной линии от
Солнца на соответствующем им радиальном расстоянии и имеют реально наблюдаемые
скорости вращения.

\item Рассмотреть движение электрона в постоянных магнитном и электрическом
полях ($\hat{B}=B_0 \hat{e_z}$, $\hat{E}=E_0 \hat{e_z}$), направленных
перендикулярно друг к другу. Сила Лоренца, действующая на заряженную частицу:
$F=q(E+\frac{1}{c}[V \times B])$, где $q$ -- заряд частицы, $V$ -- ее скорость.
\subitem Получить аналитическое выражение для координат и скоростей частицы и
построить ее траекторию для $B_0=10 \; nT, \; E_0=2 \; mV/m, \; r_0=[100,0,0] \; km, \; V_0=[100, 50, 200] \; km/s$.

\subitem Методом Рунге-Кутты построить траекторию частицы в случае, если
магнитное поле не постоянно, а задано формулой:
$\hat{B}=B_0 \tanh(x/L) \hat{e_z}$, где $B_0=10nT$, $L=1000 km$. Начальные
условия взять то же, что в предыдущем пункте. Чем качественно поведение системы
отличается в случае постоянного и переменного поля?

\item Рассмотрим простейшее одномерное уравнение диффузии функции
распределения частиц по импульсам:
\begin{equation}\label{Diffusion}
\frac{\partial f}{\partial t}=\frac{\partial}{\partial p}\left(D\frac{\partial f}{\partial p} \right)
\end{equation}

В тривиальном случае, когда $D=\mathrm{const}$, оно имеет аналитическое решение.
\subitem Получить аналитическое решение уравнения \eqref{Diffusion} в случае,
когда $D=1$, $f(t=0,p)=\delta(p)$.
\textit{Указание: для решения задачи перейти к Фурье-пространству, т.е.
рассмотреть уравнение на Фурье-образ функции распределения.}
\subitem Численно решить уравнение \eqref{Diffusion} при тех же условиях, что и в
предыдущем пункте на временном интервале (после обезразмеривания системы) $t=[0,100]$.
\subitem К решению уравнения диффузии можно подойти и с другой стороны:
при диффузии траектория каждой частицы в фазовом пространстве является случайной,
т.е. для ее описания можно использовать модель случаного блуждания. Можно показать,
что импульс каждой частицы будет являться случайной величиной и его изменение в каждый
момент времени можно описать формулой:
\begin{equation}
p(t+dt)=p(t)+\sqrt{2D}W_t
\end{equation}
где $W_t=\varepsilon \sqrt{dt}$ -- изменение Винеровского процесса,
$\varepsilon\sim N(0,1)$ -- нормально-распределенная случайная величина.

Напишите программу, которая будет моделировать случайные фазовые траектории
$\sim 10^6$ частиц и позволит проследить эволюцию функции распределения во времени.
Для этого нужно: 1) задать ансамбль частиц с импульсами, удовоетворяющими
начальному условию, 2) задать шаг по времени, 3) для каждой частицы на каждом
шаге сгенерерировать нормальное случайное число и рассчитать изменение импульса, 
4) собрать функцию распределения частиц, зная импульс каждой частицы
(бинировать импульсы и посчитать, какое количество частиц попало в каждый бин).

\subitem Сравнить все три полученных решения.


\item Промоделировать движение электрона в поле Земного магнитного диполя.
Считать, что в начальный момент времени частица находилась в точке
$r=[7 R_E,0,0]$, где $R_E$ -- радиус Земли (6371 км), имела скорость
$V=[10,20,10] km/s$. Диполь находится в начале координат, а его ось
направлена вдоль оси $z$. Какие периодические движения наблюдаются в системе?
\end{enumerate}

\end{document}
